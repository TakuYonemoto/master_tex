\documentclass[Yonemoto_master.tex]{subfiles}

\begin{document}

\chapter{まとめと展望}

\section{陽電子タイミングカウンターの位置較正}
世界最高感度で$\mu^+ \to e^+ \gamma$崩壊を探索するMEG II実験のために、512個のピクセルからなる陽電子タイミングカウンター(pixelated Timing Counter, pTC)が新しく開発された。pTCは$30$ ps台の高い時間分解能を誇り、本研究はpTCの最適な運用のために、エリアにおける構造体の位置較正及び各ピクセルの位置較正手法の開発に取り組んだ。

いずれの位置較正も数mmのオーダーの高い精度を実現し、測定や解析の手法は体系的に確立されたと言える。位置較正を適用することにより、pTCの時間分解能に対して$0.1\%$程度の向上も見込まれる。

今後、この位置較正を解析に適用することで、pTCの性能・運用は確固たるものとなるだろう。

\section{展望}
各ピクセルの位置較正における解析段階で、77個のピクセルを3Dスキャンによるデータ取得の質が低いと見なして除外してしまった。今後、再測定を行なってこれらのピクセルについても位置較正を行う必要がある。ただし、特にレーザー光を当てにくいような部分に設置されているピクセルだと考えられるため、再測定する際にも注意したい。

また、各ピクセルの位置のずれはピクセル間の飛行時間(Time of Flight, TOF)に特に影響を及ぼすと考えられるが、このTOFの見積もりはpTCの時間較正にも影響している。そのため、今後より詳細に位置較正の影響を考察していく必要がある。

さらに、陽電子スペクトロメータとしてのドリフトチェンバー(CDCH)との複合解析に備え、相対位置を較正する必要がある。これはレーザートラッカーによって測定する構造体としての相対位置の較正に加え、各ワイヤーとピクセルの相対位置を検証するような、宇宙線飛跡による位置較正が提案されている。
 


\end{document}