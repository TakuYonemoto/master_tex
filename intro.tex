\documentclass[Yonemoto_master.tex]{subfiles}
%\setcounter[chapter]{1}
\begin{document}

\chapter{序論}

素粒子物理学とは、物質を構成する最小単位「素粒子」から物理法則を統一的に理解しようとする試みである。現代素粒子物理学においては、実験的事実と良く整合する「標準模型(Standard Model, SM)」が理論的な枠組みとなる。2012年にLHCでヒッグス粒子が発見されたことで標準模型の主張は盤石なものとなったが、一方でニュートリノ振動やミューオン異常磁気能率の理論値と測定値のずれなど、標準模型では説明出来ない現象は未だ残されている。これらの実験的事実が示唆するより統一的な理論へと至るため、「標準模型を超える物理(Beyond Standard Model, BSM)」の研究が盛んに行われている。\\

標準模型においてレプトンのフレーバー混合は許されず、このことをレプトンフレーバー保存と呼ぶ。しかし、既に発見されているニュートリノ振動は中性レプトンフレーバー保存の破れを示す現象であり、「荷電レプトンフレーバー保存の破れ (charged Lepton Flavor Violation, cLFV)」 についてもその探索が進められている。\\


cLFV現象の一つである「$\mu^+ \to e^+\gamma$崩壊」について、理論と実験の両面から多くの研究がなされてきた。$\mu^+ \to e^+\gamma$崩壊の探索を目的とするMEG II実験\cite{MEGII}は、世界最高強度のミューオン源を有するスイスのポールシェラー研究所 (Paul Scherrer Institute, PSI) で間もなく稼働予定である。現行の$\mu^+ \to e^+\gamma$崩壊の分岐比上限$4.2 \times 10^{-13}$を与えた前身のMEG実験\cite{MEG}を超え、$\mathcal{O}(10^{-14})$の分岐比感度を実現するため、MEG II実験ではPSIで使用可能な最大強度のミューオンビームとそれに耐えうる高性能な検出器を運用していく。\\


MEG II実験における陽電子タイミングカウンターは、複数ヒット測定により$ 30 \rm ps$台の高い精度で陽電子の時間を測定する検出器である。「ピクセル」と呼ばれる$120 {\rm mm} \times 40 (50){\rm mm} \times 5 {\rm mm}$の小型シンチレーションカウンターが、半円筒面状のサポートの上に512個配置されたマルチピクセル構造をしている。この検出器を指して、「pixelated Timing Counter (pTC)」という呼称を用いる。\\

本論文の主題は、このpTCに対する位置較正手法を開発し、位置のずれによる影響を評価することでpTCの十全な運用を可能とすることである。\\

構成は、物理的背景(2章)、MEGII実験について(3章)、陽電子タイミングカウンターについて(4章)、位置較正について(5章)となっており、展望を交えつつ6章でまとめる。

\ifthenelse{\boolean{masterflag}}
{ }{
\begin{thebibliography}{99}
\bibitem{MEGII} A. M. Baldini et al., The design of the MEG II experiment. {\it Eur. Phys. J.}, C {\bf 78} (5), 380 (2018). \\ 
doi: \href{https://doi.org/10.1140/epjc/s10052-018-5845-6}{10.1140/epjc/s10052-018-5845-6.}
\bibitem{MEG} A. M. Baldini et al., Search for the lepton flavour violating decay $\mu^+ \to e^+ \gamma$ with the full dataset of the MEG experiment. {\it Eur. Phys. J.}, C 76 (8) (2016) 434. 
doi: https://doi.org/10.1140/epjc/s10052-016-4271-x. 
\end{thebibliography}
}

\end{document}