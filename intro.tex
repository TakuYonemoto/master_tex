\documentclass[Yonemoto_master.tex]{subfiles}
%\setcounter[chapter]{1}
\begin{document}

\chapter{序論}
素粒子物理学とは、物質を構成する最小単位から物理法則を記述する試みである。現代素粒子物理学においては、実験的事実と良く整合する『標準模型』が理論的な枠組みの基本となる。2012年にLHCでヒッグス粒子が発見され、標準模型の主張は盤石なものとなったが、未だにニュートリノ振動やミューオン異常時期能率からのずれなど、標準模型では説明の付かない実験的事実は存在する。これらを説明するため、ひいてはあらゆるエネルギー領域の物理を説明するような、『標準模型を超える物理(BSM)』の研究が盛んに行われている。
MEG実験及びその後継のMEG II実験では、標準理論を超える物理の1つである『荷電レプトンフレーバーの破れ(cLFV)』という現象のうち$\mu \to e\gamma$崩壊について探索し、BSMの手がかりを掴もうとしている。

\section{素粒子物理学におけるcLFVの探索}
\noindent $\mu \to e\gamma$ \\
$\mu \to eN$ \\
$\mu \to eee$ \\
・過去・国内外での探索

\section{MEG II実験における陽電子タイミングカウンター}
MEG実験における陽電子検出の課題として...

\section{本論文の構成について}
本論文は、物理的背景(2章)、MEGII実験における陽電子タイミングカウンターについて(3,4,5章)、位置較正についての測定・解析・結果(6,7,8章)から構成され、展望を交えつつ9章でまとめる。

\end{document}