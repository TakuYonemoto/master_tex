\documentclass[Yonemoto_master.tex]{subfiles}
%\setcounter[chapter]{1}
\begin{document}

\chapter{序論}

素粒子物理学とは、物質を構成する最小単位「素粒子」から物理法則を記述する試みである。現代素粒子物理学においては、実験的事実と良く整合する「標準模型(Standard Model)」が理論的な枠組みの基本となる。2012年にLHCでヒッグス粒子が発見されたことで標準模型の主張は盤石なものとなったが、一方でニュートリノ振動やミューオン異常時期能率からのずれなどは標準模型では説明出来ない。これらの実験的事実が示唆する「標準模型を超える物理(BSM)」の研究が盛んに行われている。

標準模型においてレプトンフレーバーは保存するとされているが、中性レプトンフレーバーの破れであるところのニュートリノ振動は既に発見されている。そして、多くのBSMは荷電レプトンフレーバーの破れ(cLFV)の存在を予言し、cLFVの実験的探索が現在活発に進められている。cLFVの一種である$\mu^+ \to e^+\gamma$崩壊についても、理論と実験の両面から多くの研究がなされてきた。\\


$\mu^+ \to e^+\gamma$崩壊の探索を目的とするMEG II実験は、世界最高強度のミューオン源を有するスイスのポールシェラー研究所(PSI)で間もなく稼働予定である。現行の$\mu^+ \to e^+\gamma$崩壊の分岐比上限$4.2 \times 10^{-13}$を与えた前身のMEG実験\cite{MEG}を超え、$\mathcal{O}(10^{-14})$の分岐比感度を実現するため、PSIで使用可能な最大強度のミューオンビームとそれに耐えうる高性能な検出器を運用していく。


MEG II実験における陽電子タイミングカウンター(pixelated Timing Counter, pTC)は、複数ヒット測定により$< 40 \rm ps$の高い精度で陽電子の時間を測定するマルチピクセル検出器である。「ピクセル」と呼ばれる$120 {\rm mm} \times 40 (50){\rm mm} \times 5 {\rm mm}$の小型シンチレーションカウンターが、半円筒面状のサポートの上に512個配置された構造をしている。\\


本論文の構成は、物理的背景(2章)、MEGII実験における陽電子タイミングカウンターについて(3,4,5章)、位置較正についての測定・解析・結果(6,7,8章)となっており、展望を交えつつ9章でまとめる。

\ifthenelse{\boolean{masterflag}}
{ }{
\begin{thebibliography}{99}
\bibitem{MEG} A. M. Baldini et al., Search for the lepton flavour violating decay $\mu^+ \to e^+ \gamma$ with the full dataset of the MEG experiment. {\it Eur. Phys. J.}, C 76 (8) (2016) 434. 
doi: 10.1140/epjc/s10052-016-4271-x. 
\end{thebibliography}
}

\end{document}