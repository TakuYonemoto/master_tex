\documentclass[report]{jsbook}
\widebaselines
\setlength{\textwidth}{\fullwidth}	
\setlength{\evensidemargin}{\oddsidemargin}
\usepackage{bm}
\usepackage[dvipdfmx]{graphicx}
\usepackage[hyphens]{url}
\usepackage[dvipdfmx]{hyperref}
\usepackage{amsmath}
\usepackage{comment}
\usepackage{subfig}
\usepackage{txfonts}
\usepackage{tabularx}
\usepackage{lscape}
\usepackage{color}
\usepackage{mathrsfs}
\usepackage{subfiles}
\usepackage{ascmac}
\usepackage{setspace}

\title{
修士論文\\[1.8cm]
{\bf MEG II実験におけるマルチピクセル陽電子タイミングカウンターの位置較正に関する研究} \\
{\LARGE Research on Position Calibration of Multi-pixelated Positron Timing Counter in MEG II Experiment}
\\[0.8cm]}
\author{東京大学大学院 理学系研究科 物理学専攻\\
素粒子物理国際研究センター 森研究室\\ \\
35-196101\\[2.5mm]
\LARGE 米本 拓}
\date{2020年 1月}

\begin{document}

\maketitle
\chapter*{概要}

標準理論を超える物理の1つである$\mu \to e\gamma$崩壊を世界最高感度で探索した国際共同実験MEGでは、崩壊分岐比に上限値$4.2 \times 10^{-13}$を与えたが発見には至らなかった。分岐比感度$\mathcal{O}(10^{-14})$を目指し$\mu \to e\gamma$崩壊の発見へと至ろうとする後継実験MEG IIのために、多数のプラスチックシンチレータとシリコン光検出器(SiPM)を搭載する新たなデザインの陽電子タイミングカウンターが製作された。\\ \\
・pTCの新たなデザインの説明 \\ \\
・位置較正について \\ \\
・成果 \\ \\

\setcounter{tocdepth}{2}
\tableofcontents

\subfile{intro.tex}
\subfile{physics.tex}
\subfile{pTC.tex}
\subfile{pos_calib.tex}
\subfile{concl.tex}

\chapter*{謝辞}

\part{付録}
\appendix
\chapter{3D測量機器について}
\chapter{軌跡再構成について}
\section{カルマンフィルター}
\section{クラスタリング}
\section{ドリフトチェンバーとのマッチング}

\begin{thebibliography}{99}
\bibitem{muon} P.A. Zyla et al. Review of Particle Physics.  {\it Progress of Theoretical and Experimental Physics}, Volume 2020, Issue 8, August 2020, 083C01. \\
doi: https://doi.org/10.1093/ptep/ptaa104
\bibitem{particles_sm} CERN website. https://home.cern/science/physics/standard-model , cited 14th December 2020.
\end{thebibliography}


\begin{comment}
・文献の引用の仕方

データは参考文献\cite{rika} にあったものを使った.
この文献\cite{ten}も参考にした。

・図の挿入の仕方
\begin{figure}[h]
\begin{center}
\includegraphics[width=7cm]{./plot1.pdf}
\caption{サイン関数のグラフ}
\end{center}
\end{figure}


\begin{verbatim}
#include <iostream>
using namespace std;
int main() {
for(int i = 1; i <= 5; i++) {
cout << "こんにちは" << i << endl;
}
return 0;
}
\end{verbatim}
\verb|\usepackage{ascmac}|して\verb|screen| 環境を使うと,枠がつきます。
\begin{screen}
\begin{verbatim}
#include <iostream>
using namespace std;
int main() {
for(int i = 1; i <= 5; i++) {
cout << "こんにちは" << i << endl;
}
return 0;
}
\end{verbatim}
\end{screen}

\begin{thebibliography}{99}
\bibitem{rika} 国立天文台編,理科年表 (丸善)
\bibitem{ten} 天文年鑑,誠文堂新光社。
\end{thebibliography}

\end{comment}

\end{document}