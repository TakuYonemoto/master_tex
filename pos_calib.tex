\documentclass[Yonemoto_master.tex]{subfiles}

\begin{document}
\chapter{陽電子タイミングカウンターの位置較正}
本章では、本研究の主題である陽電子タイミングカウンター(pixelated Timing Counter, pTC)の位置較正について、開発した手法と時間分解能に与える影響の評価を述べる。

\section{背景}
\subsection{pTCの位置測定についての二つの課題}
MEG II実験において、pTCはドリフトチェンバー(CDCH)との複合測定・解析により陽電子再構成を行う検出器である。図\ref{fig: pTC_CAD}に示されるような構造体が、図\ref{fig: double_pTC}のようにビーム上流側と下流側にターゲットを挟んで対となるように設置されるが、実験エリアにおける位置についてこれまでは測定されてこなかった。そのため、一つ目の課題として実験エリアにおけるpTCの構造体としての位置を測定し、ターゲットとの位置関係を明確にするということがあった。

\begin{figure}[h]
    \begin{tabular}{cc}
      %---- 最初の図 ---------------------------
      \begin{minipage}[t]{0.45\hsize}
        \centering
        \includegraphics[height=5cm, width=5cm]{images/USTC_overall.png}
        \caption{CADにおける上流側pTCの概観}
        \label{fig: pTC_CAD}
      \end{minipage} &
      %---- 2番目の図 --------------------------
      \begin{minipage}[t]{0.45\hsize}
        \centering
        \includegraphics[height=3cm, width=8cm]{images/track.png}
        \caption{上流と下流で対となるpTC}
        \label{fig: double_pTC}
      \end{minipage}
      %---- 図はここまで ----------------------
    \end{tabular}
  \end{figure}
加えて、複数のピクセルの位置が系統的にずれていた場合、それらは積み重なってpTCの時間再構成に影響してくる。よって二つ目の課題は、512個全てのピクセルに対する相対的な位置較正を行うことである。これは512個という対象の多さや、毎年のメンテナンスによるピクセルの交換なども考慮すると、ある程度体系化された手法の開発が求められる。

以降の節では、これら二つの課題における測定・解析について述べるが、測定に関しての手法やセットアップは2018年の段階でほぼ確立されており、本研究では解析手法の確立及びその実践・評価を主導した。

\subsection{要求される精度}
実験エリアにおけるpTCの位置測定を考える際、重要な指標となるのはターゲットからの伝搬長、すなわちCDCHで求めた飛跡長である。陽電子が入射角$45^\circ$で螺旋運動していることを仮定すれば、$z$-$\phi$面における位置測定の精度$\sigma_{z,\phi}^{(area)}$が飛跡長$l_{e^+}$に対し与える影響は、
\begin{align}
\frac{\sigma_{l_{e^+}}}{c}
= \sqrt{\left(\frac{\sigma_z^{(area)}}{c/\sqrt{2}}\right)^2 + \left(\frac{r\sigma_{\phi}^{(area)}}{c/\sqrt{2}}\right)^2 + \left(\frac{\sigma_{l_{e^+}}^{CDCH}}{c}\right)^2} 
\end{align}

と書ける。ここで、CDCHによる陽電子軌跡の不確かさ$\sigma_{l_{e^+}}^{CDCH}/c \sim 10 \ {\rm ps}$への寄与を10\%程度($\sim$1 ps)まで抑えたいので、

\begin{align}
\sqrt{\left(\frac{\sigma_z^{(area)}}{c/\sqrt{2}}\right)^2 + \left(\frac{r\sigma_{\phi}^{(area)}}{c/\sqrt{2}}\right)^2 + (10 \ {\rm ps})^2} &< (10 + 1) \ {\rm ps} \nonumber \\
\therefore \sqrt{ \left(\frac{y}{r}{\sigma_x^{(area)}}\right)^2 + \left(\frac{x}{r}{\sigma_y^{(area)}}\right)^2 + {\sigma_z^{(area)}}^2}&\lesssim 3.1 \ {\rm mm} \label{eq: 5.2}
\end{align}

\noindent を、実験エリアにおける位置測定の目標精度とする。$\left({r\sigma_{\phi}}\right)^2 = \left(\frac{y}{r}\sigma_x \right)^2 + \left(\frac{x}{r} \sigma_y\right)^2$を用いた。特に、$\left(\frac{y}{r}\sigma_x \right)^2 + \left(\frac{x}{r} \sigma_y\right)^2 < \sigma_x^2 + \sigma_y^2$より、
\begin{align} \label{eq: 5.3}
\sqrt{{\sigma_x^{(area)}}^2 + {\sigma_y^{(area)}}^2 + {\sigma_z^{(area)}}^2}  \lesssim \ 1.1 \ {\rm mm}
\end{align}
は式\ref{eq: 5.2}の十分条件となっている。

512個のピクセルに対する位置較正については、単一のピクセルの時間分解能$\sigma_{t_{e^+}}^{single} \sim 80$ psに及ぼす影響を10\%程度まで抑えることを考える。位置較正の精度を$\sigma_{z,\phi}^{(pixel)}$とすれば、上と同様にして、
\begin{align}
\sqrt{\left(\frac{\sigma_z^{(area)}}{c/\sqrt{2}}\right)^2 + \left(\frac{r\sigma_{\phi}^{(area)}}{c/\sqrt{2}}\right)^2 + (80 \ {\rm ps})^2} &< (80 + 8) \ {\rm ps} \nonumber \\
\sqrt{\left(\frac{y}{r}{\sigma_x^{(pixel)}}\right)^2 + \left(\frac{x}{r}{\sigma_y^{(pixel)}}\right)^2 + {\sigma_z^{(pixel)}}^2} &\lesssim \ 8.7 \  {\rm mm}
\end{align}

\noindent が、ピクセルに対する位置較正の目標精度となる。これは式\ref{eq: 5.3}より緩い条件なので、最終的には
\begin{align} \label{eq: 5.5}
\sqrt{{\sigma_x^{(pixel)}}^2 + {\sigma_y^{(pixel)}}^2 + {\sigma_z^{(pixel)}}^2} \lesssim \ 3.1 \ {\rm mm}
\end{align}
が各ピクセルの位置較正に対する条件である。


\section{実験エリアにおける位置測定}
\subsection{原理}
一つ目の課題である実験エリアにおける位置測定について述べる。用いるのは、レーザートラッカーと呼ばれる3次元測量装置である(図\ref{fig: lasertracker})。検出器に設置された反射鏡に向けて光を照射し、反射光の飛行時間による絶対距離測定とマイケルソン干渉計の原理(図\ref{fig: laser_principle})を用いた距離測定の両方を組み合わせることで、反射鏡の位置を精度良く捉える。表\ref{tab: laser_tracker}に、今回用いた、PSIが所有するLeica Geosystems社のレーザートラッカーについて性能を示す。

\begin{figure}[h]
    \begin{tabular}{cc}
      %---- 最初の図 ---------------------------
      \begin{minipage}[t]{0.45\hsize}
        \centering
        \includegraphics[height=7cm, width=5cm]{images/lasertracker.png}
        \caption{レーザートラッカー (Leica Absolute Tracker AT960)}
        \label{fig: lasertracker}
      \end{minipage} &
      %---- 2番目の図 --------------------------
      \begin{minipage}[t]{0.45\hsize}
        \centering
        \includegraphics[width=7cm]{images/laser_track_principle.png}
        \caption{マイケルソン干渉計の原理\cite{laser_track_principle}}
        \label{fig: laser_principle}
      \end{minipage}
      %---- 図はここまで ----------------------
    \end{tabular}
\end{figure}

\begin{table}[h]
 \centering
 \caption{レーザートラッカーの性能\cite{laser_tracker}}
 \label{tab: laser_tracker}
  \begin{tabular}{lccc}
   \hline
   モデル & AT960 \\
   測定精度 & $\pm \ 15 \ {\rm \mu m} + 6 \ {\rm \mu m / m}$ \\
   データ取得レート & 1000 points/sec \\
   \hline
  \end{tabular}
\end{table}

\subsection{2018-2020年の測定}
pTCのインストール状況に合わせ、2018年は下流側pTCに、2019年及び2020年は上流側pTCに対してMEG II実験における他の検出器と共に測定を行った。得られた結果を図\ref{fig: area_survey}に示す。2019年はCDCHに接続されるケーブルが大量に増え、11個設置するはずだったPSI所有の反射鏡(図\ref{fig: SMR})の内4個しか設置することが出来ず、十分な数を測定出来なかった。そこで我々は、2020年以降の測定のために備え付け用の反射鏡(図\ref{fig: corner1})を購入し、3個を上流側に設置した。そうしたこともあって、2020年の測定では8個の反射鏡に対し測定を行うことが出来た。

\begin{figure}[h]
    \begin{tabular}{cc}
      %---- 最初の図 ---------------------------
      \begin{minipage}[t]{0.45\hsize}
        \centering
        \includegraphics[height=7cm, width=6cm]{images/2018survey.png}
        \caption*{(a)}
      \end{minipage} &
      %---- 2番目の図 --------------------------
      \begin{minipage}[t]{0.45\hsize}
        \centering
        \includegraphics[height=7cm, width=7cm]{images/tab_2018.png}
        \caption*{(b)}
      \end{minipage}
      %---- 図はここまで ----------------------
    \end{tabular}
\begin{center}
\includegraphics[width=7cm]{images/2019survey.png}
\caption*{(c)}
\end{center}
    \begin{tabular}{cc}
      %---- 最初の図 ---------------------------
      \begin{minipage}[t]{0.45\hsize}
        \centering
        \includegraphics[height=7cm, width=6cm]{images/2020survey.png}
        \caption*{(d)}
      \end{minipage} &
      %---- 2番目の図 --------------------------
      \begin{minipage}[t]{0.45\hsize}
        \centering
        \includegraphics[height=7cm, width=7cm]{images/tab_2020.png}
        \caption*{(e)}
      \end{minipage}
      %---- 図はここまで ----------------------
    \end{tabular}
    \caption{2018-2020年における実験エリアでの位置測定。(a) 2018年(下流側pTC) (c) 2019年(上流側pTC) (d) 2020年(上流側pTC)にそれぞれ対応する。このうち2018年と2020年については十分な数のデータ点があり、(b),(e)はそれぞれの実験エリア外での事前測定との各反射鏡の位置を比較した値である。}
    \label{fig: area_survey}
\end{figure}

\begin{figure}[h]
    \begin{tabular}{cc}
      %---- 最初の図 ---------------------------
      \begin{minipage}[t]{0.45\hsize}
        \centering
        \includegraphics[height=5.5cm, width=4cm]{images/CornerCube.jpg}
        \caption{PSI所有の反射鏡、Spherical mounted Retro-reflector (SMR)}
         \label{fig: SMR}
      \end{minipage} &
      %---- 2番目の図 --------------------------
      \begin{minipage}[t]{0.45\hsize}
        \centering
        \includegraphics[height=5.5cm, width=7cm]{images/corner1.JPG}
        \caption{購入したニューポート社製の反射鏡、\cite{newport}}
         \label{fig: corner1}
      \end{minipage}
      %---- 図はここまで ----------------------
    \end{tabular}
\end{figure}

\subsection{評価}
エリア外での事前測定との相対位置のズレは、pTC構造体の変形の影響も受けるが、この位置測定の手法による不確かさを内包しているとも考えられる。
ここで、図\ref{fig: area_survey}(b),(e)より、上流側・下流側pTCに対し取得した全データ点で式\ref{eq: 5.3}が成立しているため、要求される精度でエリアでの設置位置が測定できたと言える。

\clearpage

\section{3Dスキャンによる位置較正}
本研究の要となる、3Dスキャナーを用いた位置構成について述べる。

\subsection{3Dスキャンにおける測量}
3Dスキャナーは、三角測距方式のレーザー測定器である。物体にレーザーを照射し、物体の位置によって反射光の位相が異なることを利用して物体の3次元位置を精度良く測定する器具である。使用したFARO社のEdge ScanArm HD(図\ref{fig: FARO})について、その性能を表\ref{tab: FARO_spec}に記す。
\begin{figure}[h]
  \centering
  \includegraphics[keepaspectratio, scale=0.7]{images/FARO.png}
  \caption{使用した3Dスキャナー (FARO Edge ScanArm HD)}
  \label{fig: FARO}
\end{figure}

\begin{table}[h]
 \centering
 \caption{3Dスキャナーの性能 \cite{FARO}}
 \label{tab: FARO_spec}
  \begin{tabular}{lccc}
   \hline
   モデル & FARO Edge ScanArm HD \\
   測定精度 & $\pm \ 25 \ \mu m$ \\
   データ取得レート & 560000 points/sec \\
   \hline
  \end{tabular}
\end{table}


\subsection{試験測定・解析}

はじめ、図\ref{fig: first_scan}に示す三個のピクセルに対し試験的な3Dスキャンを行なった。これらのデータに対し詳細な解析やフィッティングを行い、512ピクセルの位置を求める手法への応用を検討していた。

\begin{figure}[h]
    \begin{tabular}{cc}
      %---- 最初の図 ---------------------------
      \begin{minipage}[t]{0.45\hsize}
        \centering
        \includegraphics[keepaspectratio, scale=0.6]{images/first_threes.png}
        \caption*{(a)}
      \end{minipage} &
      %---- 2番目の図 --------------------------
      \begin{minipage}[t]{0.45\hsize}
        \centering
        \includegraphics[keepaspectratio, scale=0.6]{images/first_scan.png}
        \caption*{(b)}
      \end{minipage}
      %---- 図はここまで ----------------------
    \end{tabular}
    \caption{三個のピクセルに対する試験的な3Dスキャン。(a): 実際の写真、(b): 3Dスキャンにより取得したデータ}
    \label{fig: first_scan}
  \end{figure}
 
図\ref{fig: first_scan}(b)の最も奥のピクセルは特に3Dスキャンの状態が良かったため、これに対し一通りの解析を試してみることから始めた。まずは大まかにこのピクセルのみを切り分け、図\ref{fig: first_fit} (左)のようなデータ点群を得た。そしてそれらに対し、プラスチックシンチレータの寸法と同じ$120 {\rm mm} \times 40 (50){\rm mm} \times 5 {\rm mm}$の直方体の最近接面との距離の二乗の総和を最小化するように、回転及び並行移動をパラメータとしてフィットしたものが図\ref{fig: first_fit}(右)である。これについて、フィットが収束した場合、その際のフィッティングパラメータから元の座標及びピクセルの回転方向が分かる効果的な位置測定方法であることが分かった。
 
\begin{figure}
    \centering
    \includegraphics[keepaspectratio, scale=0.4]{images/first_onefit.png}
    \caption{直方体フィッティングによる試験的な位置測定。左: 図1.9(b)から切り分けたデータ点群、右: 切り分けたデータ点群をプラスチックシンチレータと同じ寸法の直方体面でフィットし、各辺が座標軸に平行になるように移動した結果}
    \label{fig: first_fit}
\end{figure}
  
\subsection{本測定}
 2018年度パイロットランの終了後、pTC全体に対して初めての3Dスキャンを行なった。スキャンデータの概観として、図\ref{fig: US_scan}に3Dスキャナーにより上流側pTCに対し取得したデータ点群を描画したものを示す。前節のレーザートラッカーによる実験エリアでの位置測定と照合するため、レーザートラッカー用のSMRを模した球体をレーザートラッカーによる測定と同じ位置に置いてスキャンしている。

\begin{figure}[h]
\begin{center}
\includegraphics[width=7cm]{images/US_scan.png}
\caption{pTC(上流側)の3Dスキャンデータ}
\label{fig: US_scan}
\end{center}
\end{figure}


\subsection{スキャンデータの解析}

  当初は、全データ点群から各ピクセルを切り分け、各々に対しピクセルを模した直方体によるフィッティングを行うことで位置を較正していくことを想定していた。しかし、いくつかのピクセルについて個別にデータ点を見てみると、それぞれの取得状況には図\ref{fig: good_bad}に示されるようにかなりの質の差があった。

\begin{figure}[h]
    \begin{tabular}{cc}
      %---- 最初の図 ---------------------------
      \begin{minipage}[t]{0.45\hsize}
        \centering
        \includegraphics[width=7cm]{images/good.png}
        \caption*{(a)}
      \end{minipage} &
      %---- 2番目の図 --------------------------
      \begin{minipage}[t]{0.45\hsize}
        \centering
        \includegraphics[width=7cm]{images/bad.png}
         \caption*{(b)}
      \end{minipage}
      %---- 図はここまで ----------------------
    \end{tabular}
    \caption{図1.11から切り分けた単一ピクセル。(a): 比較的3Dスキャンのムラが少なく、多くのデータ点が含まれているピクセル、(b): レーザー光の届かない場所が多く、データ点の少ないピクセル}
    \label{fig: good_bad}
  \end{figure}

構造体の大きさも相まって、内側のピクセルには3Dスキャナーがアクセスしづらかったためか、図\ref{fig: good_bad}(b)のように少ないデータ点しかスキャンされていないピクセルが散見された。図\ref{fig: good_bad}(a)は全ピクセル中でもかなり綺麗にデータが取れているピクセルであったが、そのようなピクセルですら3Dスキャナーの入射光の限界から六面のうち二面は欠損していた。よって前々節で述べた直方体によるフィッティングの手法は、これらのピクセルの位置測定に適しておらず、新たな手法を開発する必要があった。

\begin{itemize}
\item 手作業での初期値設定+小規模なフィッティング
\end{itemize}
初期段階における解析の様子を、図\ref{fig: manual}に示す。面の数を減らした直方体でのフィッティングを行うことを考え、回転における不定性を無くすためにある程度まで手で初期位置と回転を与え、その後フィッティングを行うことで位置と回転を測定しようとした。これを48個のピクセルに対し行なったが、非常に効率の悪い手法であった。最終的なピクセルの位置と回転を求めるためには、二回の並行移動+回転のパラメータ($x_1,y_1,z_1,\theta_{x_1},\theta_{y_1},\theta_{z_1}$)($x_2,y_2,z_2,\theta_{x_2},\theta_{y_2},\theta_{z_2}$)を一組のパラメータ($x,y,z,\theta_{x},\theta_{y},\theta_{z}$)で表す必要があり、この手法を完成させることは出来なかった。

\begin{figure}[h]
\begin{center}
\includegraphics[width=12cm]{images/manual_ana.png}
\caption{初期段階での3Dスキャンデータに対する解析。値を手で与えながら大まかな位置と回転角を合わせ(Pre-trans)、特定の領域内のデータ点(緑色)について三面のみの直方体(赤色)とのパラメータの範囲を狭めてフィッティングを行う。}
\label{fig: manual}
\end{center}
\end{figure}

\begin{itemize}
\item CADデータにおける設計値からの初期値の取得
\end{itemize}
次に、初期値をCADデータにおける設計値から得られないかを考えた。当初からこれを行わなかったのは、CADデータにおける座標系と3Dスキャンデータにおける座標系を一致させる方法が無かったからである。しかしCADデータにおける座標系とは、前節のレーザートラッカーによる実験エリアでの位置測定の座標系とほぼ同一のものである。2019年パイロットランの段階で、pTCの実験エリアでの位置を上流側・下流側ともにレーザートラッカーで高精度で測定出来たため、レーザートラッカーでのSMRの位置測定を3DスキャンによるSMRの位置測定と照合することで、二つの座標系を一致させることが出来るようになった。

これを用いて、全てのピクセルに対し位置と回転の初期値を与えることが出来た。図\ref{fig: scan_pixel}にその様子を示す。多くのピクセルに対し上面のデータ点(赤色)はレーザー光がよく届いたためか他の面に比べて綺麗に取れていたため、これを位置測定に用いることにした。上面が明らかに軸に沿っていないような、さらなる回転を必要とするピクセルは見られなかったため、先に述べた並行移動+回転を二セット行う際のパラメータのまとめ辛さを鑑みて、以降は並行移動のみを行い、ピクセルの中心位置の設計値からのずれを求めることにした。
\begin{figure}[h]
  \centering
  \includegraphics[width=9cm]{images/pixel1.png}
  \caption{3Dスキャンデータにおけるピクセルに対し、設計値から位置と回転の初期値を与えたもの。灰: データ点のうち位置測定に用いなかった点、赤: データ点のうち位置測定に用いたピクセルの上面に相当する点、緑: 赤で示した点を用いて求めた上面の中心を通る断面、青: プラスチックシンチレータと遮光カバーの設計上の位置。上面と見なしたのは$x \in [-22, -19 {\rm mm}]$であるような点であり、明らかに側面を含んでしまう場合は個別に調整した。}
  \label{fig: scan_pixel}
\end{figure}

以下、ピクセルの高さ、厚さ、幅の方向をそれぞれ$x, y, z$とする。次に、データ点の$y, z$座標についても制限を付け、データ点の選別を行う。図\ref{fig: scan_pixel}において上面として選んだデータ点の中で、$y, z$座標の最大値と最小値に注目し、
\begin{align}
y_{width} = | y_{max} - y_{min} | \\
z_{width} = | z_{max} - z_{min} |
\end{align}
と置いた$y, z$方向への面の幅について、図\ref{fig: yz_select}にプロットを示す。平均値からの距離が標準偏差の2倍よりも大きいデータについては、上面のスキャン状態が悪いと見なし512個中77個のピクセルを以降の解析から除外した。

\begin{figure}[h]
  \centering
  \includegraphics[width=9cm]{images/yz_select.png}
  \caption{$y, z$方向における面の幅についてのプロット。軸に垂直に描いた青線はそれぞれの平均値を表し、平均値からの距離が標準偏差の2倍以下の領域を緑の四角で囲んだ。緑の四角に含まれないピクセルを、上面のスキャン状態が悪いものと見なして除外した。赤で反射材とカバーを除いた部分から計算される設計値を示した。}
  \label{fig: yz_select}
\end{figure}

画像では伝わりにくいが、データ点には疎になっている部分と密になっている部分があり、選別をかけたデータ点に対しても、その全てを平均して上面の中心を求めることは適さないようだった。そこで、$y, z$座標の最大値と最小値の差が一定の領域に含まれるデータのみを選別したことを踏まえ、$y, z$座標の最大値と最小値の平均から上面の中心を求めることが適していると考えた。それを計算し図示したものが図\ref{fig: scan_pixel}における緑色の線である。

以上により求めたピクセル上面の中心座標を、ピクセルの中心位置の$y, z$座標とする。最後に$x$方向について議論する。スキャンデータから分かる通り、各ピクセルの遮光用のカバーは曲面を形成しており側面が定義しづらい。しかし上面はほぼ$y$軸に沿っており、経験的にも上面はプラスチックシンチレータに沿ってカバーがきちんと張られていると考えられた。最大3mmの広がりを持つ$x \in [-22, -19 {\rm \ mm}]$の点に対して平均を取り、そこから$x$方向にプラスチックシンチレータの高さ40mmの半分、20mmを加えたものをピクセルの中心位置のx座標とした。

\subsection{結果}
前小節で求めた各ピクセルの中心座標について、設計値からのずれをプロットしたものを図\ref{fig: result}に示す。
\begin{figure}[h]
    \begin{tabular}{ccc}
      %---- 最初の図 ---------------------------
      \begin{minipage}[t]{0.30\hsize}
        \centering
        \includegraphics[height=4cm, width=5cm]{images/us_x_res.png}
        \caption*{(a)}
      \end{minipage} &
      %---- 2番目の図 --------------------------
      \begin{minipage}[t]{0.30\hsize}
        \centering
        \includegraphics[height=4cm, width=5cm]{images/us_y_res.png}
         \caption*{(b)}
      \end{minipage} &
      \begin{minipage}[t]{0.30\hsize}
        \centering
        \includegraphics[height=4cm, width=5cm]{images/us_z_res.png}
         \caption*{(c)}
      \end{minipage} \\ \\
      %---- 最初の図 ---------------------------
      \begin{minipage}[t]{0.30\hsize}
        \centering
        \includegraphics[height=4cm, width=5cm]{images/ds_x_res.png}
        \caption*{(d)}
      \end{minipage} &
      %---- 2番目の図 --------------------------
      \begin{minipage}[t]{0.30\hsize}
        \centering
        \includegraphics[height=4cm, width=5cm]{images/ds_y_res.png}
         \caption*{(e)}
      \end{minipage} &
      \begin{minipage}[t]{0.30\hsize}
        \centering
        \includegraphics[height=4cm, width=5cm]{images/ds_z_res.png}
         \caption*{(f)}
      \end{minipage} \\
      %---- 図はここまで ----------------------
    \end{tabular}
    \caption{各ピクセルの中心位置の設計値からのずれの分布。(a),(b),(c) 上流側スキャンデータにおけるx,y,z座標、(d),(e),(f) 下流側スキャンデータにおけるx,y,z座標に対応。横軸の単位はいずれもmm。}
    \label{fig: result}
  \end{figure}

\subsection{考察}

中心位置のずれにおける標準偏差は、真のずれと3Dスキャナーを用いたピクセルに対する位置較正手法及びレーザートラッカーによる測量の不確かさが合わさったものである。よって$\sigma_{x,y,z}^{(pixel)}$に対し、$\sigma_{x,y,z}^{(pixel)} < 1.2$ [mm]であると言えるため、
\begin{align}
\sqrt{{\sigma_x^{(pixel)}}^2 + {\sigma_y^{(pixel)}}^2 + {\sigma_z^{(pixel)}}^2} \lesssim  2.07 \ {\rm mm}
\end{align}
であることから、この位置較正は式\ref{eq: 5.5}の要求を満たす。

\section{シミュレーションによる時間分解能への影響の検証}
前節までで議論した位置較正について、従来のシミュレーションと位置較正を適用したシミュレーションを比較することにより、時間分解能に与える影響を検証する。図\ref{fig: TOF_res}に、各セットアップにおけるピクセル間飛行時間(TOF)の見積もりと真値とのずれをプロットした。

\begin{figure}[h]
    \begin{tabular}{cc}
      %---- 最初の図 ---------------------------
      \begin{minipage}[t]{0.45\hsize}
        \centering
        \includegraphics[width=7cm]{images/hist_std.png}
        \caption*{(a)}
      \end{minipage} &
      %---- 2番目の図 --------------------------
      \begin{minipage}[t]{0.45\hsize}
        \centering
        \includegraphics[width=7cm]{images/hist_survey.png}
         \caption*{(b)}
      \end{minipage}
      %---- 図はここまで ----------------------
    \end{tabular}
    \caption{モンテカルロシミュレーションによるTOFの見積もりと真値とのずれ。(a) 従来のセットアップ、 (b) 位置較正の結果を適用したセットアップ。(b)では解析に位置較正の情報を適用した場合(青)と適用しない場合(赤)を載せる。}
    \label{fig: TOF_res}
 \end{figure}

この結果から、解析段階で位置較正を適用した場合、TOFの見積もりに対し$10\%$程度の精度向上が見られる。

シミュレーションにおけるpTCのヒット数$N_{\rm hit}$の時間測定は、
\begin{align}
t_{e^+} (N_{\rm hit}) = \frac{1}{N_{\rm hit}} \sum_{i_{\rm hit}}^{N_{\rm hit}} (t_{i_{\rm hit}}^{rec} - TOF_{1 \to i_{\rm hit}}^{rec} - t_1^{MC})
\end{align}

で行われ、その時間分解能は、
\begin{align}
\sigma_{t_{e^+}} (N_{\rm hit}) = \sqrt{\frac{\sigma_{single}^2 + \sigma_{TOF}^2}{N_{\rm hit}} + \sigma_{other}^2}
\end{align}

の形で書ける。この式における$\sigma_{TOF}^2$が$10\%$向上するとすれば、$\sigma_{t_{e^+}}$は$0.1\%$程度向上すると見積もれる。($\sigma_{t_{e^+}}^{single} \sim 80$ psを用いた)

\end{document}