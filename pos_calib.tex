\documentclass[Yonemoto_master.tex]{subfiles}

\begin{document}
\chapter{陽電子タイミングカウンターの位置較正}
本章では、本研究の主題である陽電子タイミングカウンターの位置較正について、開発した手法と時間分解能に与える影響の評価を述べる。

\begin{figure}[h]
\begin{center}
\includegraphics[width=7cm]{images/USTC_overall.png}
\caption{CADにおけるpTC(上流側)の概観}
\end{center}
\end{figure}

\section{3Dスキャンによる位置較正}
\begin{figure}[h]
\begin{center}
\includegraphics[width=7cm]{images/US_scan.png}
\caption{pTC(上流側)の3Dスキャンデータ}
\end{center}
\end{figure}


\subsection{3Dスキャンにおける測量}
\subsection{実験エリアにおける測量}
\subsection{測量基準点 (reference point)}
\subsection{スキャンデータの解析}
\subsection{結果}
\subsection{考察}

\section{軌跡再構成による位置較正の試み}
\subsection{原理}
\subsection{課題}
\subsection{考察}

\section{各ピクセルの位置のずれが与える影響の評価}


\end{document}