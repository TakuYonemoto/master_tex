\documentclass[Yonemoto_master.tex]{subfiles}

\begin{document}
\chapter{陽電子タイミングカウンターの位置較正}
本章では、本研究の主題である陽電子タイミングカウンター(pixelated Timing Counter, pTC)の位置較正について、開発した手法と時間分解能に与える影響の評価を述べる。

\begin{figure}[h]
\begin{center}
\includegraphics[width=7cm]{images/USTC_overall.png}
\caption{CADにおけるpTC(上流側)の概観}
\label{fig: pTC_CAD}
\end{center}
\end{figure}

\section{背景}
MEG II実験において、pTCはドリフトチェンバー(CDCH)との陽電子の複合測定・解析により陽電子再構成を行う検出器である。図\ref{fig: pTC_CAD}に示されるような構造体が、ビーム上流側と下流側にターゲットを挟んで対となるように設置されるが、実験エリアにおける位置についてこれまでは検証されてこなかった。そのため、一つ目の課題として実験エリアにおけるグローバルな位置を測定し、CDCHとの位置関係を明確にするということがあった。

加えて、複数ヒットによる時間測定を行うために、複数のピクセルの位置が系統的にずれていた場合、それは積み重なって時間再構成に影響してくる。よって、二つ目の課題として512個全てのピクセルに対し、各々の時間分解能を考慮し、時間測定に影響を及ぼさないような精度でのローカルな位置の較正を行うことがあった。

\section{3Dスキャンによる位置較正}
本研究の要となる、3Dスキャナーを用いた位置構成について述べる。

\subsection{3Dスキャンにおける測量}
・光学的な原理など
\subsection{スキャンデータの解析}
\begin{figure}[h]
\begin{center}
\includegraphics[width=7cm]{images/US_scan.png}
\caption{pTC(上流側)の3Dスキャンデータ}
\end{center}
\end{figure}
3Dスキャンにおける座標系での測定結果を、CAD上の設計値と比較するために、実験エリアにおける座標系に移す必要がある。
\subsection{実験エリアにおける測量}
ここで、一つ目の課題であった実験エリアにおけるグローバルな位置較正と、前節の3Dスキャンから得られた結果をCAD上の設計値と比較するために必要な、実験エリアにおける測量について述べる。
\subsection{測量基準点 (reference point)}
\subsection{結果}
\subsection{考察}

\section{軌跡再構成による位置較正の試み}
\subsection{原理}
\subsection{課題}
\subsection{考察}

\section{各ピクセルの位置のずれが与える影響の評価}


\end{document}