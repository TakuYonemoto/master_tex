\documentclass[Yonemoto_master.tex]{subfiles}
%\setcounter[chapter]{2}
\begin{document}

\chapter{$\mu \to e \gamma$ 崩壊}

\section{標準理論において}
ミューオンの基本性質を以下にまとめる。

\begin{table}[h]
 \label{table:Muon_prop}
 \centering
  \begin{tabular}{cc}
   \hline
   質量 & 105.6583745 $\pm$ 0.0000024 MeV \\
   寿命 & (2.1969811 $\pm$ 0.0000022 $\times 10^{-6}$) s \\ 
   \hline
  \end{tabular}
 \caption{ミューオンの性質 \cite{muon}}
\end{table}
   
標準理論において、


\begin{figure}[h]
\begin{center}
\includegraphics[width=10cm]{./images/particles_sm.png}
\caption{標準理論における素粒子 \cite{particles_sm}}
\end{center}
\end{figure}

\section{標準理論を超える物理において}
\section{実験における信号・背景事象}

\chapter{MEG II 実験}

\section{MEG 実験}
\section{MEG II 実験}
\subsection{MEG実験からのアップグレード}
\subsection{ドリフトチェンバー (CDCH)}
\subsection{陽電子タイミングカウンター (pTC)}
\subsection{液体キセノンガンマ線検出器 (LXe)}
\subsection{輻射崩壊検出器 (RDC)}
\subsection{DAQ}
\subsection{展望}



\end{document}